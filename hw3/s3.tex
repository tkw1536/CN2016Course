\documentclass[a4paper,10pt]{article}

\usepackage[utf8]{inputenc}

\usepackage{amsfonts}
\usepackage{amssymb}
\usepackage{amsmath}
\usepackage{amsthm}
\usepackage{graphicx}
\usepackage{listings}
\usepackage{hyperref}
\usepackage{a4wide}
\usepackage[show]{ed}

\usepackage[english]{babel}

\title{Computer Networks Fall 2016\\Problem Sheet \#3}
\author{Tom Wiesing}
\date{\today}

\begin{document}
\maketitle

\section{Problem 1}
\subsection{1a)}
We assign \textbf{R1} IP address 198.51.100.1 on interface en1.1 and 203.0.113.1 on en1.2. Furthermore we configure the following forwarding table: \\

\begin{tabular}{ l | l | l}
  Prefix & Next Hop & Interface \\
  \hline
  198.51.100.3/32 & 198.51.100.3 & en1.1 \\
  203.0.113.4/32 & 203.0.113.4 & en1.2 \\
  \hline
\end{tabular}

We assign \textbf{R2} IP address 198.51.100.2 on interface en2.1 and 203.0.113.2 on en2.2. Furthermore we configure the following forwarding table:

\begin{tabular}{ l | l | l}
  Prefix & Next Hop & Interface \\
  \hline
  198.51.100.3/32 & 198.51.100.3 & en2.1 \\
  203.0.113.4/32 & 203.0.113.4 & en2.2 \\
  \hline
\end{tabular}

\subsection{1b)}

In this scenario the following messages will be sent:
\begin{enumerate}
  \item A sends a TCP SYN from A to B.
  \begin{itemize}
    \item A makes an ARP query for the R1 ip adresss on en1.1 (198.51.100.1)
    \item R1 responds with an ARP response $\text{mac}(en2.1)$
    \item R1 makes an ARP request to the B ip adress on enB.0 (203.0.113.4)
    \item B responds with an ARP response $\text{mac}(enB.0)$
  \end{itemize}
  \item B sends a TCP SYN-ACK from B to A.
  \begin{itemize}
    \item B makes an ARP request for the R2 ip adresss on en2.2 (203.0.113.2)
    \item R2 responds with an ARP response $\text{mac}(en2.2)$
    \item R2 makes an ARP request to the A ip adress on enA.0 (198.51.100.3)
    \item A responds with an ARP response $\text{mac}(enA.0)$
  \end{itemize}
  \item A now sends a TCP ACK from A to B.
  \item This procedure can then repeat.
\end{enumerate}

\begin{tabular}{ l | l | l | l | l | l | l }
  no & segments & eth-src & eth-dst & ip-src & ip-dest & description \\\hline

  % TCP SYN A -> B
  1 & A, B, \dots & $\text{mac}(ena.0)$ & $\text{mac}()$ & $\text{ip}(ena.0)$ & $\text{ip}(en1.1)$ & ARP request $\text{ip}(en1.1)$  \\
  2 & B, A, \dots & $\text{mac}(en1.1)$ & $\text{mac}(ena.0)$ & $\text{ip}(en1.1)$ & $\text{ip}(ena.0)$ & ARP response $\text{ip}(en1.1)$  \\
  3 & A, B & $\text{mac}(ena.0)$ & $\text{mac}(enb.0)$ & ip(ena.0) & $\text{ip}(en1.1)$ & TCP SYN A $\rightarrow$ B (step 1)\\
  4 & D, F, \dots & $\text{mac}(en1.2)$ & $\text{mac}()$ & $\text{ip}(en1.1)$ & $\text{ip}(enb.0)$ & ARP request $\text{ip}(enb.0)$)  \\
  5 & F, D, \dots & $\text{mac}(enb.0)$ & $\text{mac}(en1.2)$ & $\text{ip}(enb.0)$ & $\text{ip}(en1.1)$ & ARP response $\text{ip}(enb.0)$)  \\
  6 & D, F & $\text{mac}(en1.2)$ & $\text{mac}(enb.0)$  & $\text{ip}(ena.0)$ & $\text{ip}(enb.0)$ & TCP SYN A $\rightarrow$ B (step 2)\\\hline

  % TCP SYN-ACK B -> A
  7 & F, E, \dots & $\text{mac}(enb.0)$ & $\text{mac}()$ & $\text{ip}(enb.0)$ & $\text{ip}(en2.2)$ & ARP request $\text{ip}(en2.2)$  \\
  8 & E, F, \dots & $\text{mac}(en2.2)$ & $\text{mac}(enb.0)$ & $\text{ip}(en2.2)$ & $\text{ip}(enb.0)$ & ARP response $\text{ip}(en2.2)$  \\
  9 & E, F & $\text{mac}(enb.0)$ & $\text{mac}(en2.2)$ & $\text{ip}(enb.0)$ & $\text{ip}(ena.0)$ & TCP SYN-ACK B $\rightarrow$ A (step 1)  \\
  10 & C, A, \dots & $\text{mac}(en2.1)$ & $\text{mac}()$ & $\text{ip}(en2.1)$ & $\text{ip}(ena.0)$ & ARP request $\text{ip}(ena.0)$ \\
  11 & A, C, \dots & $\text{mac}(ena.0)$ & $\text{mac}(en2.1)$ & $\text{ip}(ena.0)$ & $\text{ip}(en2.1)$ & ARP response $\text{ip}(ena.0)$  \\
  12 & C, A & $\text{mac}(en2.1)$ & $\text{mac}(ena.0)$ & $\text{ip}(enb.0)$ & $\text{ip}(ena.0)$ & TCP SYN-ACK B $\rightarrow$ A (step 2) \\\hline

  % TCP ACK A -> B
  13 & A, B & $\text{mac}(ena.0)$ & $\text{mac}(en1.1)$ & $\text{ip}(ena.0)$ & $\text{ip}(enb.0)$ & TCP ACK A $\rightarrow$ B (step 1)\\
  14 & D, F & $\text{mac}(en1.2)$ & $\text{mac}(enb.0)$  & $\text{ip}(ena.0)$ & $\text{ip}(enb.0)$ & TCP ACK A $\rightarrow$ B (step 2)\\\hline

  \hline
\end{tabular}

\subsection{1c)}
Benefits:
\begin{itemize}
  \item it is easy to configure
  \item traffic between A and B and B and A goes via different routes $\rightarrow$ each segment has to handle less traffic
\end{itemize}
Problems:
\begin{itemize}
  \item The routes can \textbf{only} reach A and B. If more nodes are connected, the network has to be reconfigured.
\end{itemize}

\section{Problem 2}

\subsection{2a)}
The longest prefix length in the tables is 2, so it is sufficient to check all prefixes of length 2 to check equivalence of $F_1$ and $F_2$. In $F_1$, we have
\begin{itemize}
  \item $00* \rightarrow R_2$
  \item $01* \rightarrow R_1$
  \item $10* \rightarrow R_2$
  \item $11* \rightarrow R_3$
\end{itemize}
and in $F_2$ we have
\begin{itemize}
  \item $00* \rightarrow R_2$
  \item $01* \rightarrow R_1$
  \item $10* \rightarrow R_2$
  \item $11* \rightarrow R_3$
\end{itemize}
Thus they are equivalent.

\subsection{2b)}
There is no equivalent forwarding forwarding table with less than four entries. As there are three different hosts involved in the routes, at least 3 entries are needed. Thus if there is a table with exactly 3 entires. But it is not possible to combine the two entries for $R_2$.
\end{document}
