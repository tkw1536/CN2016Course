\documentclass[a4paper,10pt]{article}

\usepackage[utf8]{inputenc}

\usepackage{amsfonts}
\usepackage{amssymb}
\usepackage{amsmath}
\usepackage{amsthm}
\usepackage{graphicx}
\usepackage{listings}
\usepackage{hyperref}
\usepackage[show]{ed}

\usepackage[english]{babel}

\title{Computer Networks Fall 2016\\Problem Sheet \#1}
\author{Tom Wiesing}
\date{\today}

\begin{document}
\maketitle

\section{Problem 1.1: optical fibre transmission}

You are deploying an optical fiber across Germany with a length of $l = 790 \mathrm{km}$. The optical fiber has a data rate of $r = 10 \mathrm{Gbps} = 10^{10} \mathrm{bps}$. The signal propagation speed in the optical fiber is $v = 2 {\cdot} 10^8 \frac{\mathrm{m}}{\mathrm{s}}$.

\subsection{a) How many bits can be simultaneously in transit on the wire?}

A bit exits transit after travelling at most the length of the entire fibre. Any bit takes at most $\tau = \frac{l}{v} = \frac{79 {\cdot} 10^4 \mathrm{m}}{2 {\cdot} 10^8 \frac{\mathrm{m}}{\mathrm{s}}} = 3.95 {\cdot} 10^{-3} \mathrm{s}$ to do this. A situation where the maximal amount of bits are in transit simultaneously occurs if the fibre constantly transmits bits at the maximal data rate during this time. Thus at most
$r {\cdot} \tau = 10^{10} \mathrm{bps} {\cdot} 3.95 {\cdot} 10^{-3} \mathrm{s} = 3.95 {\cdot} 10^7 \mathrm{b}$ can be in transit at the same time.

\subsection{b) What is the round-trip-time imposed by the propagation delay of the wire?}

We have already computed the propagation delay $\tau$ above to be $3.95 {\cdot} 10^{-3} \mathrm{s}$. The round-trip-time is $2 \tau = 7.9 {\cdot} 10^{-3} \mathrm{s}$.

\subsection{c) Two former Jacobs Students are now located in Berlin and Seattle. They both measure the round-trip-time to Google's DNS server (IPv4 address 8.8.8.8). The person located in Berlin measures a minimum round-trip-time $rtt_{B} = 6 \mathrm{ms}$ while the person in Seatlle measures  a minimum round-trip-time of $rtt_{S} = 12 \mathrm{ms}$. Are they sending packets to the same server? }

The available information is insufficient to determine if the are sending packets to the same server, although it is likely that they are not.

If the persons are sending packets to the same server, then packet travelling from Seattle to Berlin would take an average of $\frac{1}{2}(rtt_{B} + rtt_{S}) = 9 \mathrm{ms}$ to arrive. Since Seattle and Berlin have a distance of $8141 \mathrm{km}$, so if there was an optical fiber of the same specification as above connecting the two locations this distance would need at minimum $\frac{v}{8141 \mathrm{km}} = 40.71 \mathrm{ms}$. That means the cable is not fast enough to deliver packages at that speed.

Usually a single server is responding to each IPv4 adress, but according to the Google DNS FAQ\footnote{\url{https://developers.google.com/speed/public-dns/faq\#anycast}} Google uses AnyCast to route to the geograpically nearest server. Since Berlin and Seattle are seperated by a significant distance, the persons will likely send packages to different servers.

\section{Problem 1.2: crc checksums}

A transmission link uses a cyclic redundancy check (CRC) code to detect transmission errors. The generator polynomial is $G(x) = x^4 + x^2 + 1$. The bit sequence $1101 0110 0101 1010$ has been received by a network interface card. Does the message pass the CRC check? Show why or why not.

In binary, the generator corresponds to $10101$. We perform a CRC check below.
\lstinputlisting{crc-hw.txt}
\noindent As we get the non-zero remainder $1110$, the bit sequence does not pass the CRC check.

\section{Problem 1.3: ARQ protocols}

Automatic Repeat reQuest (ARQ) is an error-control method for data transmission that uses acknowledgements and timeouts to achieve reliable data transmission over an unreliable service. If the sender does not receive an acknowledgment before the timeout, it usually re-transmits the frame until the sender receives an acknowledgment or exceeds a predefined number of retransmissions.

Well-known types of ARQ protocols are
\begin{itemize}
  \item Stop-and-wait ARQ,
  \item Go-Back-N ARQ, and
  \item Selective Repeat ARQ.
\end{itemize}
Read about these three well-known ARQ protocols and briefly summarize how they work.

\subsection{Stop-and-wait ARQ}
In this protocol data frames are sent from sender to receiver one by one. Each frame has an internal CRC check. Upon receiving a frame, the receiver sends back an ACK (acknowledgement) packet if the CRC check passes. If the CRC check does not pass, or the frame is lost along the way, no such package is sent.
The sender expects to receive an ACK package for each frame it sends. If it does not receive such a confirmation after a fixed timeout, it re-sends the data frame. The name of this protocol originates from this behaviour, the sender stops and waits for the acknoledgement.

\subsection{Go-Back-N ARQ}
Contrary to the stop-and-wait ARQ above, packages are not send one-by-one.

In this protcol, each frame has a continuous sequence number. This is used by the receiver to keep track of the number of frames received. Upon receiving a packet with the expected sequence number, the server sends an acknowledgement for it. If it receives a broken, or out-of-order packet, it instead sends an acknowledgement for the last correctly sent packet.
The sender sends $N$ sequential packages. After sending all of them, it can then detect the last successfull ACK it received and start sending again from that point onwards.

\subsection{Selective Repeat ARQ}
Compared to the Go-Back-N protocol above, in this protocol the sender only resends those frames for which is has not received an acknolegement. The receiver has to store all packages, even if they are out-of-order, and re-assemble them in the correct order afterwards.

\section{Problem 1.4: stop-and-wait ARQ efficiency}

Given a fixed frame size of $n_f$ bits with an overhead (header, checksums) of $n_o$ bits, an acknowledgment size of $n_a$ bits, the propagation delay $t_p$, the frame transmission delay $t_f$ , the acknowledgment transmission delay $t_a$ and the processing delay $t_c$ (for both frames and acknowledgments), and the data rate $r$, what is the time $t$ that passes from the start of the frame transmission until the acknowledgment has been processed? Assume that no transmission errors occur.

The effective data rate $r_e$ is the amount of information bits transmitted per time unit, i.e., the useful information transmitted without any overhead bits or acknowledgment bits. Derive an expression for the effective data rate $r_e$ and the efficiency $e$, which is obtained by normalizing the effective data rate $r_e$ by the data rate r, i.e. $e = \frac{r_e}{r}$.

In stop-and-go-ARQ, we first need to send the data-frame with overhead, i.e. a total of $n = n_o + n_f$ bits. This first has a frame transmission delay of $t_f$ and then at a data rate of $r$, needs $\frac{n}{r} = \frac{n_o + n_f}{r}$ seconds to be fully transmitted and a time of  $t_p$ to reach the destination. We then have a a processing delay of $t_c$. Thus to successfully send the original frame from sender to receiver we need $t_f + \frac{n_o + n_f}{r} + t_p + t_c$. Next we send back the acknowledgement (size $n_a$)  with transmission delay $t_a$ for $\frac{n_a}{r}$  with a propagnation delay of $t_p$. We then again have a processing delay of $t_c$. For sending the ACK we thus need $t_a + \frac{n_a}{r} + t_p + t_c$.
Overall this gives a time of
$$ t = t_f + t_a + 2 t_p + 2 t_c + \frac{n_o + n_f + n_a}{r}$$
from the start of the frame transmission until the acknowledgment has been processed.

To compute the effective data rate $r_e$, we notice that only $n_f$ pieces of data have been transmitted in time $t$. Hence
$$r_e = \frac{n_f}{t} = \frac{n_f}{t_f + t_a + 2 t_p + 2 t_c + \frac{n_o + n_f + n_a}{r}}$$
. The efficiency $e$ is given by
$$e = \frac{r_e}{r} = \frac{\frac{n_f}{t_f + t_a + 2 t_p + 2 t_c + \frac{n_o + n_f + n_a}{r}}}{r} = \frac{n_f}{r \left( t_f + t_a + 2 t_p + 2 t_c \right) + n_o + n_f + n_a}$$
\end{document}
