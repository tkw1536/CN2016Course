\documentclass[a4paper,10pt]{article}

\usepackage[utf8]{inputenc}

\usepackage{amsfonts}
\usepackage{amssymb}
\usepackage{amsmath}
\usepackage{amsthm}
\usepackage{graphicx}
\usepackage{listings}
\usepackage{hyperref}
\usepackage{a4wide}
\usepackage[show]{ed}

\usepackage[english]{babel}

\title{Computer Networks Fall 2016\\Problem Sheet \#4}
\author{Tom Wiesing}
\date{\today}

\begin{document}
\maketitle

\section{Problem 1}
The following systems and interfaces are being simulated:

\begin{tabular}{ l | l | l | l}
  System & Interface & IPv4 address & Description \\
  \hline
  h1 & h1-eth0 & 10.0.0.1 & Ethernet connecting h1 to s1 \\
  \hline
  h2 & h2-eth0 & 10.0.0.2 & Ethernet connecting h2 to s1 \\
  \hline
  s1 & s1 & (none) & Unconnected ethernet interface of switch s1\\
  s1 & s1-eth0 & 192.168.101.15 & Ethernet connecting S1 to NATed interface with host machine\\
  s1 & s1-eth1 & (none) & Ethernet connecting s1 to h1\\
  s1 & s1-eth2 & (none) & Ethernet connecting s1 to h2\\
  \hline
  (all) & lo & 127.0.0.1 & Local loopback interface connecting each node to itself\\
  \hline
\end{tabular}

\section{Problem 2}

The following results were achieved with iperf:

\begin{tabular}{ l | l | l}
  Link config & Transfer & Data Rate \\
  \hline
  & 1.32 Gbits/sec & 1.32 Gbits/sec\\
  \texttt{bw=10} & 9.57 Mbits/sec & 10.6 Mbits/sec \\
  \texttt{bw=10,delay='10ms'} & 9.51 Mbits/sec & 11.5 Mbits/sec \\
  \texttt{bw=10,delay='10ms',loss=1} & 3.30 Mbits/sec & 3.60 Mbits/sec \\
  \texttt{bw=10,delay='10ms',loss=5} & 168 Kbits/sec & 197 Kbits/sec \\
  \texttt{bw=10,delay='10ms',loss=10} & 27.0 Kbits/sec & 39.4 Kbits/sec \\
  \hline
\end{tabular}

\section{Problem 3}

We use the following Python script to configure and test the required topology:

\lstinputlisting[language=Python]{s4-p3.py}

\end{document}
